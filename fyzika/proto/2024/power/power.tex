\documentclass{article}
\usepackage[slovak]{babel}
\usepackage{graphicx}
\begin{document}
\title {\textbf{Laboratórny protokol - Meranie Vlastného výkonu}}
\author{Adam Jenča}
\setlength{\tabcolsep}{18pt}
\renewcommand{\arraystretch}{1.5}
\maketitle
\newpage
\section{Úvod}
V tomto experimente sme mali zistiť svoj vlastný výkon počas pokojnej chôdze a behu hore schodmi.
Výkon $P$ je definovaný ako $\frac{W}{t}$, kde $W$ je práca vykonaná za určitý čas $t$.
Práca je zadefinovaná ako $F\cdot s$, kde $F$ je sila, ktorá pôsobí na určitej dráhe $s$.
\section{Pomôcky a postup}
\subsection{Pomôcky}
\begin{itemize}
	\item Váha
	\item Stopky
	\item Pásmové meradlo
\end{itemize}
\subsection{Postup}
\begin{enumerate}
	\item Odmeriame výšku schoda $h_s$ pre 3 rôzne schody. Tieto výšky spriemerujeme do $\overline{h_s}$.
	\item Zistíme a zapíšeme si počet schodov $n_s$ na určitom úseku schodiska, na ktorom budeme experiment vykonávať.
	\item Zmeriame a zapíšeme svoju hmotnosť $m$.
	\item Odmeriame a zapíšeme svoj čas pri chôdzi($t_c$) a pri behu($t_b$)
\end{enumerate}
\section{Údaje a výpočty}
Zmerali sme nasledujúce údaje:
$\overline{h_s}=14,5\ cm$\\
$n_s=48$\\
$m=50 kg$\\
$t_c=16,49s$\\
$t_b=9,95s$\\
\subsection{Výpočty}
V tomto prípade pôsobíme proti tiažovej sile ($F_G$). Tiažovú silu vypočítame z hmotnosti ako $F_G = m\cdot g$, kde $g\approx 9,81 \frac{N}{kg}$.
Keďže $m=50\ kg$, $F_G = 9,81\ \frac{N}{kg}\cdot 50\ kg = 490,5\ N$.\\
Dráhu $s$, na ktorej pôsobíme proti $F_G$ (teda celkovú prekonanú výšku) vypočítame jednoducho ako $n_s\cdot\overline{h_s} = 48 \cdot 14,5\ cm = 696\ cm = 6,96\ m$.
Vykonanú prácu vypočítame ako $W = F_G\cdot s = 490,5\ N \cdot 6,96\ m = 3413,88\ J$\\
Nakoniec spočítame svoj výkon pri chôdzi $P_c = \frac{W}{t_c} = \frac{3413,88\ J}{16,49\ s}=\mathbf{207,027 W}$\\
Podobne: $P_b = \frac{W}{t_b} = \frac{3413,88\ J}{9,95\ s} = \mathbf{343,105 W}$\\
\section{Záver}
Cieľom experimentu bolo zmerať svoj vlastný výkon počas behu a chôdze hore schodmi. Experiment vyšiel pomerne dobre, ale niektoré veci neboli úplne najlepšie:
\begin{itemize}
	\item Počas cesty hore sme museli prekonávať istú treciu silu oproti schodom. Toto mohlo spôsobiť zvýšenie výkonu oproti nameranej hodnote.
	\item Úsek schodov nebol súvislý, museli sme prekonávať medziposchodia.
	\item Meranie časov mohlo byť nepresné kvôli nedokonalému reakčného času merateľa.
	\item Úsek schodov mohol mať inú výšku ako $n_s\cdot\overline{h_s}$, ak by niektoré schody boli výrazne odlišné od priemeru.
\end{itemize}
Vyšli nasledovné údaje:
Výkon pri behu $P_b = \mathbf{343,105 W}$\\
Výkon pri chôdzi $P_c = \mathbf{207,027 W}$
\end{document}
