\documentclass{article}
\usepackage[margin=0.5in]{geometry}
\usepackage{hyperref}
\usepackage{polyglossia}
\setdefaultlanguage{slovak}
\begin{document}
\begin{center} 
	\huge \textbf{Meranie priemernej teploty v jednom týždni \\ v decembri} 
\end{center}
\normalsize
\vskip 5cm
\section{Teoretický Úvod}
\textbf{Teplota} je fyzikálna veličina, ktorá opisuje energiu spôsobenú pohybom molekúl.
Meria sa v troch jednotkách : $^{\circ}$C (stupne Celsia), $^{\circ}$F (stupne Fahrenheita) a $^{\circ}$K (stupne Kelvina). Stupne Celsia sa používajú najmä v Európe, stupne Fahrenheita najmä v Amerike\footnote{\href{https://worldpopulationreview.com/country-rankings/countries-that-use-fahrenheit}{Všetky krajiny používajúce stupne Fahrenheita}} a stupne Kelvina sa používaju pri vedeckých výskumoch
\section{Pomôcky}
V ideálnom experimente by mal byť použitý \textbf{teplomer}.
Táto pomôcka ale nebola k dispozícii.
Meranie teploty bolo spravené pomocou počítačového programu napísaného experimentátorom
\end{document}
