\documentclass{article}
\usepackage[slovak]{babel}
\usepackage[top=1in, bottom=1.25in, left=1.25in, right=1.25in]{geometry}
\begin{document}
\noindent
\large
Adam Jenča\\
Tercia A\\
SŠ Novohradská, Bratislava\\
Príklad Z9-I-4\\
\vskip 10mm \noindent
Aby sme dostali najmenší možný výsledok, musia $a$ aj $b$ byť v tvare
$7^x.11^y$, pretože konštantné členy v rovnici sú iba 7 a 11, a treba upraviť ich exponenty na rovnaké.\\
Skúsme zistiť, aké najmenšie môžu $x$ a $y$ byť pre $a$.

Keďže $7a^3=11b^5$, vieme, že $3x+1$ musí byť deliteľné piatimi, pretože $7.(7^{3x})$ musí byť piata mocnina.\\
Toto platí pre najmenšie $x = 3$,kde    ($3x+1 = 10$).\\
Vieme aj, že $3y-1$ musí byť deliteľné piatimi, pretože $\frac{11^{3y}}{11}$ musí byť piata mocnina.\\
Najmenšie také $y$ je $2$,kde ($3y-1 = 5$).\\
$a$ je teda rovné $\mathbf{7^3.11^2 = 41503}$.\\
Teraz zistíme to isté pre $b$.
Vieme, že v tomto prípade musí byť $5x - 1$ deliteľné 3 , pretože $\frac{11^{5x}}{11}$ musí byť tretia mocnina.\\
Najmenšie také $x$ je $2$, kde ($5x-1 = 9$).\\
$5y+1$ musí byť tiež deliteľné tromi, pretože $11.{11^{5x}}$ musí byť tretia mocnina.\\
Najmenšie také $y$ je $1$, kde ($5y+1 = 6$)\\
$b$ je teda rovné $\mathbf{7^2 . 11 = 539}$\\
Teraz si to overíme.
Na ľavej strane máme $7a^3 = 7.(7^3.11^2)^3=7.7^9.11^{6}=7^{10}.11^{6}$\\
Na pravej strane je  $11b^5 = 11.(11.7^2)^5=11.11^5.7^{10}=7^{10}.11^{6}$\\
Ako vidno, obe strany sú rovnaké. Naše riešenie je najmenšie, pretože akékoľvek nižšie by nemohli zmeniť exponenty na rovnaké.

\end{document}
