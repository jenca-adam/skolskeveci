\documentclass{article}
\usepackage[slovak]{babel}
\usepackage[top=1in,left=3cm, right=3cm]{geometry}
\begin{document}
\large
\noindent
Adam Jenča\\
Tercia A\\
SŠ Novohradská, Bratislava\\
Príklad Z9-I-3\\
\vskip 10mm \noindent
Najprv skúsme prvý prípad, v ktorom chceme dostať päticu trojok.\\
Prvé číslo (3) nemusíme meniť. \\
Druhé (8) najprv podelíme 2 a potom odčítame 1. Kúzelníka 1 a 2 sme použili raz, takže nám pri oboch zostávajú ešte 4 použitia. \\
Od tretieho (9) najprv odčítame 1 a potom postupujeme rovnako ako pri osmičke. Kúzelníka 1 sme teraz použili celkom 3-krát (zostáva 2) a kúzelníka 2 2-krát(zostáva 3).\\
Štvrté (2) vynásobíme tromi a podelíme dvomi. Kúzelníka 1 sme použili spolu 3-krát (zost. 2), kúzelníka 2 rovnako, a kúzelníka 3 raz (zost. 4).\\
Od piateho (4) iba odčítame 1, na čo nám zvyšné použitia kúzelníka 1 bohate postačia.\\
\textbf{Prvý prípad je teda možný}\\
Pri druhom prípade máme dostať číslo 5.
Toto číslo sa dá dostať dvomi spôsobmi: odčítaním 1 od 6 , a podelením 10 dvomi.
Číslo desať ale nemáme ako dostať, pretože je najvyššie možné, a preto jediný spôsob je zväčšiť nejaké číslo. To sa ale nedá, pretože čísla môžeme zväčšovať iba ich vynásobením tromi, a 10 nie je deliteľné 3.\\
Zostáva nám už iba odčítanie 1 od 6, ktoré musíme vykonať pri každom čísle.\\
Keďže čísel, ktoré máme zmeniť, je 5, musíme odčítanie 1(kúz. č. 1) použiť 5-krát.\\
Z toho vyplýva, že ho pri žiadnej zmene čísla nemôžeme použiť viac než raz.\\
To ale nesedí, pretože pri čísle 9 musíme použiť odčítanie aspoň 2-krát -- raz ako prvú operáciu, pretože nič iné nemôžeme na 9 použiť, a raz pri zmene 6 na 5.\\
\textbf{Druhý prípad je z toho dôvodu nemožný}



\end{document}
