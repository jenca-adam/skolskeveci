\documentclass{article}
\usepackage[slovak]{babel}
\usepackage{amsfonts}
\newcommand \nat{%
	\mathbb{N}
	}
\newcommand \innat{%
	\in\nat
	}
\usepackage[top=1in, bottom=1.25in, left=1.25in, right=1.25in]{geometry}
\begin{document}
\noindent
\large
Adam Jenča\\
Tercia A\\
SŠ Novohradská, Bratislava\\
Príklad Z9-I-1\\
\vskip 10mm \noindent
Každý člen prvej postupnosti $a_n$ vyzerá takto:\\

{ $a_n = 2023 + nd$, kde $d$ je diferencia prvej postupnosti.}\\
\noindent
V druhej je to podobne:\\

{$b_n = 2023 + ne$, kde $e$ je diferencia druhej postupnosti.}\\\noindent
Teraz od oboch postupností odčítame 2023, aby sa začínali v nule, na výsledku to nič nezmení.\\


\noindent
Označme si postupnosť spoločných čísel $c$.
Pre každý člen postupnosti $c_i$ platí že 
$$
c_i = kd = ne ; \{k,n\}\subseteq \nat
$$
$c$ je aritmetická postupnosť, pretože keď sa dostaneme k prvému členu, $a_k$ a $b_n$  sú rovnaké, a teda môžeme postupnosti upraviť odčítaním $a_k$ zase na nuly.\\
Označme si jej diferenciu $f$\\
$c$ má 26 členov medzi 0 a 1000.\\
Keď nerátame nulu, má 25 členov od 1 po 1000.\\
$f$ bude teda $1000:25=40$.
Označme si koeficient $d$ pri prvom člene $c$ $k_0$ a v tej istej situácii koeficient $e$ $n_0$\\
Prvý prvok $c$ okrem nuly bude $$c_1=1f=d. k_0=e. n_0=40$$\\
Vieme, že $d$ a $e$ sú v pomere $5:2$, teda $\frac{d}{e} = \frac{5}{2}$.\\
To si upravíme cez $5d = 2e$ na $e=\frac{5}{2}d$,
Vieme preto, že $$d. k_0 = \frac{5d}{2}n_0 = 40$$.
Vynásobíme si všetko dvomi.
$$2d. k_0 = 5d. n_0 = 80$$
Vyberieme si odtiaľ rovnicu $2d. k_0 = 5d. n_0$.
Teraz podelíme obe strany $d$:
$$2k_0 = 5n_0$$
Keďže $k_0 \innat$ aj $n_0\innat$, môžeme povedať, že $$2k_0 = 5n_0 = 10x; x\innat$$
Pretože $40$ je najmenšie spoločné číslo, musia $k_0$ a $n_0$ byť najmenšie čísla spĺňajúce rovnicu vyššie.\\
Najmenšie možné $x = 1$. Teda $2k_0 = 5n_0 = 10$.
Preto $k_0 = 5$ a $n_0 = 2$.
Vieme, že 
$$d. k_0 = 5d = 40$$
\noindent Preto
$$d = \frac{40}{5} =\mathbf{8}$$
\noindent
a $$e = \frac{5}{2}d=\frac{5}{2}. 8 = \frac{40}{2}=\mathbf{20}$$
\noindent
Rozdiel diferencií $\Delta_d=e-d=\mathbf{12}$.


\end{document}
