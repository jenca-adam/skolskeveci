\documentclass{article}
\usepackage[slovak]{babel}
\begin{document}
\noindent
Adam Jenča\\
Tercia A\\
SŠ Novohradská, Bratislava\\
Príklad 1\\
\vskip 10mm \noindent
Zapíšme si zadanie ako rovnice:\\
Prvá:
$$
\frac{\frac{a + b}{2} + \frac{b + c}{2}}{2} = \frac{a+b+c}{3}
$$
Druhá:
$$
\frac{a + c}{2} = 2022
$$
Začnime s prvou rovnicou.
Najprv si zjednodušíme zložený zlomok na ľavej strane.
$$
\frac{\frac{a + b}{2} + \frac{b + c}{2}}{2} = \frac{\frac{a+2b+c}{2}}{2} = \frac{a+2b+c}{4}
$$
Teraz upravíme rovnicu už zo zjednodušeným zlomkom.
$$
\frac{a+2b+c}{4} =  \frac{a+b+c}{3}
$$
Dáme si zlomky na spoločného menovateľa:
$$
\frac{3a+6b+3c}{12} = \frac{4a+4b+4c}{12}
$$
Vynásobíme obidve strany 12:
$$
3a+6b+3c = 4a+4b+4c
$$
Odčítame od oboch strán $3a+4b+3c$
$$
2b = a+c
$$
Podelíme obe strany 2
$$
b = \frac{a+c}{2}
$$
Z druhej rovnice vieme, že 
$$
b = \frac{a+c}{2} = 2022
$$
Teda:
$$
a+b+c = b + (a+c) = b + 2b = 3b = 3 \cdot 2022 = 6066
$$


\end{document}
