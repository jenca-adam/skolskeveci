\documentclass{article}
\usepackage[slovak]{babel}
\usepackage[top=1in, bottom=1.25in, left=1.25in, right=1.25in]{geometry}
\begin{document}
\large
\noindent
Adam Jenča\\
Tercia A\\
SŠ Novohradská, Bratislava\\
Príklad Z9-I-5\\
\vskip 10mm \noindent
Označme si sny $x$,ilúzie $y$, šlofíky $a$ a nočné mory $b$.
Podľa prvého cestovateľa vieme určiť,že
$$
	4x = 7y + 2a + b
$$
Keďže nás $a$ ani $b$ nezaujímajú, môžeme ich z tejto rovnice odstrániť tak, že si $2a+b$ označíme $c$, čo nám uľahčí ďalší postup.
Rovnica teraz vyzerá takto:
$$
	4x = 7y + c
$$
Teraz pripočítame k obom stranám $-c-4x$, a dostaneme
$$
	-c = 7y - 4x
$$
Po vynásobení oboch strán rovnice $-1$ nám vyjde
$$
	c = 4x - 7y
$$
Vezmime si druhú rovnicu:
$$
	7x = 4y + 4a + 2b
$$
Podobným trikom ako minule sa zbavíme $a$ a $b$
$$
	7x = 4y + 2c
$$
Teraz si za $c$ dosadíme $4x-7y$, a vyjde nám
$$
	7x = 4y + 8x - 14y
$$
z čoho úpravou pravej strany dostaneme
$$
	7x = 8x - 10y
$$
Teraz k obom stranám pripočítame $10y - 7x$, a tu je výsledok:
$$
	10y = x
$$
\textbf{Za jeden sen možno teda kúpiť 10 ilúzií.}

\end{document}
