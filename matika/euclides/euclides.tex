\documentclass{article}
\usepackage{amsmath}
\usepackage{polyglossia}
\setdefaultlanguage{slovak}
\title{Euklidov dôkaz nekonečnosti prvočísel}
\author{Adam Jenča}
\usepackage{amsthm}
\begin{document}
{{\maketitle}}
Dnes vieme, že prvočísel je nekonečne mnoho.
Ako prvý to dokázal starogrécky matematik Euklides.
Predpokladajme, že prvočísel je konečne veľa.
Nech P je množina všetkých prvočísel $P=\{p_1,p_2,...,p_n\}$ .

Potom dokážeme získať prvočíslo, ktoré nie je v množine tak, že spravíme súčin všetkých prvočísel v P a prirátame k nemu jednotku:
\[
p_{n+1}= \prod_{i=1}^n p_i+1
\]
\noindent
Toto číslo je prvočíslo, lebo po delení hociktorým $p \in P$ dáva zvyšok 1 a určite nie je v množine P.\\
Týmto sme dokázali, že prvočísel je nekonečne mnoho, lebo nikdy nevieme zostaviť množinu všetkých prvočísel, ktorá má konečnú veľkosť.
\end{document}
