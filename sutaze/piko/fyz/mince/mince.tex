\documentclass{article}
\usepackage{array}
\usepackage{bm}
\begin{document}
\begin{center}
	\huge \textbf{Krtince z mincí}\\
	\large \texttt{Príklad 4}\\
	\large \texttt{Adam Jenča}\\
	\large \texttt{Sekunda A}\\
	\large \texttt{SŠ Novohradská 3, Bratislava}\\
\end{center}
\section{Úvod}
Mojim cieľom bolo zistiť, koľko mincí treba na stlačenie klávesy. 
Používal som klávesy enter, pravý shift a medzerník s 1-centovými mincami.
\section{Postup}
Pre každú klávesu:
\begin{enumerate}
\item Opakujeme 5-krát:
\begin{enumerate}
\item Kladieme mince na klávesu až kým sa nestlačí. Pozorne počítame mince.
\item Keď sa klávesa pod váhou mincí stlačí, zapíšeme si počet mincí $M_n$ 
\end{enumerate}
\item Vypočítame priemer $\overline{M}$ pomocou vzorca 
	\[
		\overline{M}=\frac{\displaystyle{\sum_{i=1}^5 M_i}}{5}=\frac{M_1+M_2+\dots+M_5}{5} 
	\]
\item Priemer zapíšeme do tabuľky
\end{enumerate}
\section{Tabuľky}
\subsection{Klávesa Enter}
\begin{tabular}{|c|c|>{\bfseries}c|}
	\hline
	Číslo merania&Počet mincí&\normalfont{Priemer po $n$-tom meraní}\\
	\hline
	$1$&$22$&22\\
	\hline
	$2$&$25$&23.5\\
	\hline
	$3$&$26$& 24.$\bm {\overline{3}}$\\
	\hline
	$4$&$29$&25.5\\
	\hline
	$5$&$28$&26\\
	\hline
\end{tabular}\\
Pre klávesu enter nám vyšiel priemer $\bm {26}$
\subsection{Klávesa Shift}
\begin{tabular}{|c|c|>{\bfseries}c|}
	\hline
	Číslo merania&Počet mincí&\normalfont{Priemer po $n$-tom meraní}\\
	\hline
	$1$&$32$&32\\
	\hline
	$2$&$25$&28.5\\
	\hline
	$3$&$26$& 27.$\bm {\overline{6}}$\\
	\hline
	$4$&$29$&25.5\\
	\hline
	$5$&$28$&26\\
	\hline
\end{tabular}\\
Pre klávesu enter nám vyšiel priemer $\bm {26}$

\end{document}
