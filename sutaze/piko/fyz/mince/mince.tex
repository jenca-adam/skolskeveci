\documentclass{article}
\usepackage{array}
\usepackage{bm}
\begin{document}
\begin{center}
	\huge \textbf{Krtince z mincí}\\
	\large \texttt{Príklad 4}\\
	\large \texttt{Adam Jenča}\\
	\large \texttt{Sekunda A}\\
	\large \texttt{SŠ Novohradská 3, Bratislava}\\
\end{center}
\section{Úvod}
Mojim cieľom bolo zistiť, koľko mincí treba na stlačenie klávesy. 
Používal som klávesy enter, pravý shift a medzerník s 1-centovými mincami.
\section{Postup}
Pre každú klávesu:
\begin{enumerate}
\item Opakujeme 5-krát:
\begin{enumerate}
\item Kladieme mince na klávesu až kým sa nestlačí. Pozorne počítame mince.
\item Keď sa klávesa pod váhou mincí stlačí, zapíšeme si počet mincí $M_n$
\item Vypočítame priemer po $n$-tom meraní pomocou vzorca
	\[
		\overline{M_n}=\frac{\displaystyle{\sum_{i=1}^N M_i}} = \frac{M_1+M_2+\dots+M_n}{n}
	\]
\end{enumerate}
\item Vypočítame priemer $\overline{M}$ pomocou vzorca 
	\[
		\overline{M}=\frac{\displaystyle{\sum_{i=1}^5 M_i}}{5}=\frac{M_1+M_2+\dots+M_5}{5} 
	\]
\item Priemer zapíšeme do tabuľky
\end{enumerate}
\section{Tabuľky}
\subsection{Klávesa Enter}
\begin{tabular}{|c|c|>{\bfseries}c|}
	\hline
	Číslo merania&Počet mincí&\normalfont{Priemer po $n$-tom meraní}\\
	\hline
	$1$&$22$&22\\
	\hline
	$2$&$25$&23.5\\
	\hline
	$3$&$26$& 24.$\bm {\overline{3}}$\\
	\hline
	$4$&$29$&25.5\\
	\hline
	$5$&$28$&26\\
	\hline
\end{tabular}\\
Pre klávesu \textbf{Enter} nám vyšiel priemer $\bm {26}$
\subsection{Klávesa Ľavý Shift}
\begin{tabular}{|c|c|>{\bfseries}c|}
	\hline
	Číslo merania&Počet mincí&\normalfont{Priemer po $n$-tom meraní}\\
	\hline
	$1$&$32$&32\\
	\hline
	$2$&$27$&29.5\\
	\hline
	$3$&$30$& 29.$\bm {\overline{6}}$\\
	\hline
	$4$&$29$&29.5\\
	\hline
	$5$&$23$&28.2\\
	\hline
\end{tabular}\\
Pre klávesu \textbf {Shift} nám vyšiel priemer $\bm {28.2}$
\subsection{Klávesa Medzerník}
\begin{tabular}{|c|c|>{\bfseries}c|}
	\hline
	Číslo merania&Počet mincí&\normalfont{Priemer po $n$-tom meraní}\\
	\hline
	$1$&$20$&20\\
	\hline
	$2$&$30$&25\\
	\hline
	$3$&$29$& 26.$\bm {\overline{3}}$\\
	\hline
	$4$&$30$&27.25\\
	\hline
	$5$&$22$&26.2\\
	\hline
\end{tabular}\\
Pre klávesu \textbf {Medzerník} nám vyšiel priemer $\bm {26.2}$
\subsection{Všetky klávesy}
\begin{tabular}{|c|>{\bfseries}c|}
	\hline
	Klávesa&\normalfont{Priemerný počet mincí}\\
	\hline
	Enter&26\\
	\hline
	Ľavý Shift&26.2\\
	\hline
	Medzerník&28.2\\
	\hline
\end{tabular}
\subsection{Celkový priemer}
	\[
		\overline{c} = \frac{\displaystyle \sum ^{15}_{i=0}V_i}{15} = \frac{22+25+26+29+28+32+27+30+29+23+20+30+29+30+22}{15} = 26.8%UF 
	\]
	kde $V$ je n-tica všetkých meraní.
\section{Záver}
Vyšlo mi, že na stlačenie priemernej klávesy treba $26.8$ jednocentovej mince.
\textbf{Čo som mohol zlepšiť?}\\
Mohol som použiť mince s väčšou hodnotou ako 1-centovou (napríklad 10-centové) aby som nestaval také obrovské pyramídy, že s pomaly nezmestia na klávesu.
Mohol som experiment vyskúšať na rôznych klávesniciach rôznych typov a výsledok spriemerovať pre dosiahnutie všeobecnejšieho výsledku
\end{document}
