\documentclass{article}
\usepackage{tikz}
\usepackage[slovak]{babel}
\begin{document}
\begin{center}
	\huge \textbf{Kartový zázrak}\\
	\large \texttt{Príklad 4}\\
	\large \texttt{Adam Jenča}\\
	\large \texttt{Sekunda A}\\
	\large \texttt{SŠ Novohradská 3, Bratislava}\\
\end{center}
\vskip 1cm
\begin{quote}
Eliška a Vínomaš majú pred sebou kôpku s 12 kartami a idú hrať hru. Striedavo si z kôpky berú dve alebo tri karty. Vínomaš začína. Ak na konci zostane 1 karta, je to remíza, inak vyhráva ten, kto zobral poslednú kartu. Vínomaš zistil, že môže brať karty tak, že určite vyhrá, bez ohľadu na Eliškine ťahy.
	\textit{\textbf{Koľko kariet si musí Vínomaš potiahnuť v prvom ťahu a ako má hrať ďalšie ťahy, aby si vedel zaručiť, že si zoberie aj poslednú kartu?}}
\end{quote}
Povedzme, že by Vínomaš ťahal 3 karty.
Potom by na kôpke zostalo 9 kariet.
\begin{center}
\begin{tikzpicture}
	\node[draw] at (4,0.5) {12};
	\draw  (4,0) -- (6,-2);
	\node[draw] at (5.5,-0.75) {3};
	\node[draw] at (6,-2.5) {9};
\end{tikzpicture}
\end{center}
Povedzme, že teraz Eliška vezme 3 karty.

\end{document}
